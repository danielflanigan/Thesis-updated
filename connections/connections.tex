\chapter{Connections to other work}
\label{chp:connections}

This appendix is intended to facilitate comparison between the results presented here and other works on quasiparticle dynamics.
In a quasiparticle number model like the one used here, all of the dynamical results can be derived from the rate equation for the quasiparticle density.
Table~\ref{tab:connections} summarizes the equivalences between variables that can be inferred by comparing the rate equations given here.
\textcite{Zmuidzinas2012ARCMP} gives for the recombination rate
\begin{equation}
\Gamma_r
  =
  \frac{N_{qp}^2}{2 N^* \tau_\mathrm{max}}
  + \frac{N_{qp}}{\tau_\mathrm{max}},
\end{equation}
where $N_{qp}$ is the quasiparticle number and $N^*$ and $\tau_\mathrm{max}$ are constants.
For bare recombination, \textcite{Wilson2004PRB} give
\begin{equation}
\dv{N}{t}
  =
  2 \left( \Gamma_G - \frac{1}{2} \frac{R}{\mathrm{vol}} N^2 \right),
\end{equation}
where $\Gamma_G$ is the generation \textit{event} rate, $R$ is the recombination constant and $N$ is the number of quasiparticles.
When they include the phonon system, they derive a modified equation that includes an effective recombination constant
$R^* = R / F_\omega$, equivalent to $\qprecombinationeff$ here.
\textcite{Wang2014NatComm} use
\begin{equation}
\dv{x_{qp}}{t}
 =
 -r x_{qp}^2 - s x_{qp} + g,
\end{equation}
where $x_{qp} = n_{qp} / n_{cp}$, $n_{qp}$ is the density of quasiparticles and $n_{cp}$ is the density of Cooper pairs.
The number of Cooper pairs is not given explicitly but can be inferred to be equal to
$n_{cp} = \ssdos \gap_\zerotemp$ in the notation used here
by comparing their expression for the recombination constant $r$ with Equation~\ref{eqn:qprecombination} for $\qprecombinationeff$.
Their solution to their rate equation is equivalent to the solution to Equation~\ref{eqn:rate_qpdensity} given here.
Writing the equation in terms of perturbations to the steady-state density, as is done here, results in a simpler solution and a straightforward treatment of small perturbations.

\begin{table}[tbp]
\centering
\caption
{Connections between notation used in this thesis and in other works.}
\renewcommand{\arraystretch}{1.2}
\begin{tabular}{cccl}
\toprule
This thesis & \textcite{Zmuidzinas2012ARCMP} & \textcite{Wilson2004PRB} & \textcite{Wang2014NatComm} \\
\midrule
$\qprelaxationtime$ & $\tau_\mathrm{qp}$ & $\tau_r^*$ & $\tau_\mathrm{ss}$ \\
$\qprecombinationeff$ & $(2 n^* \tau_\mathrm{max})^{-1}$ & $R^*$ & $n_{cp} r$ \\
$\qpsingledecay$ & $\tau_\mathrm{max}^{-1}$ & $\Gamma_t$ & $s$ \\
$\ssdos$ & $N_0$ & $D(\varepsilon_F) / 2$ &  \\
$\phonontrapping$ &  & $F_\omega$ & $F$ \\
\bottomrule
\end{tabular}
\label{tab:connections}
\end{table}
