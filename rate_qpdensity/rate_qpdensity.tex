\chapter{Solutions of the rate equation for the quasiparticle density}
\label{chp:rate_qpdensity}

\todo[inline]{Derive solutions of Equation~\ref{eqn:rate_qpdensity}.}

Soon after a large initial perturbation satisfying
$\qprecombinationeff \qprelaxationtime \, \delta\qpdensity(0) \gg 1$
(such a perturbation must be positive) the solution to first order in $\time / \qprelaxationtime$ is
\begin{equation}
\delta\qpdensity(\time)
  =
  \frac{\delta\qpdensity(0)}{1 + \qprecombinationeff \, \delta\qpdensity(0) \, \time}.
\end{equation}
When measuring the energy of single energetic photons or , the goal is to measure $\delta\qpnumber(0)$, which is presumably proportional to the deposited energy in the quasiparticle system and thus, on average, to the particle energy.
A given initial perturbation has an associated crossover time $\time_\mathrm{x} = 1 / \qprecombinationeff \delta\qpdensity(0)$.
The larger the initial perturbation to the density, the shorter the crossover time.
A measurement of the quasiparticle system between the down-conversion process has finished~\autocite{Kozorezov2000PRB} and the crossover time 

At this time the density crosses over to a trajectory
$\delta\qpdensity(\time) \approx 1 / \qprecombinationeff \time$
that is approximately independent of the initial
The more diluted the excess quasiparticles, the slower they will decay.

However, as pointed out by \textcite{Rothwarf1967PRL}, at very early times $\time \ll \qprecombinationtime$ we may have to take into account the fact that the population of energetic recombination phonons is still developing.
The quantity $[\qprecombinationeff \, \delta\qpdensity(0)]^{-1}$ is a time scale for the peak height to be observable.
After this, the behavior for all large perturbations follows a universal trajectory $\delta\qpdensity(\time) \approx (\qprecombinationeff \time)^{-1}$, independent of the both size of the initial perturbation and the steady-state density.
Later, at $\time \sim \qprelaxationtime$, the behavior crosses over to exponential decay. 