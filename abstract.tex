Kinetic inductance detectors (KIDs) are superconducting thin-film microresonators that are sensitive photon detectors.
These detectors are a candidate for the next generation of experiments designed to measure the polarization of the cosmic microwave background (CMB).
I discuss the basic theory needed to understand the response of a KID to light, focusing on the dynamics of the quasiparticle system.
I derive an equation that describes the dynamics of the quasiparticle number, solve it in a simplified form not previously published, and show that it can describe the dynamic response of a detector. 
Magnetic flux vortices in a superconducting thin film can be a significant source of dissipation, and I demonstrate some techniques to prevent their formation.
Based on the presented theory, I derive a corrected version of a widely-used equation for the quasiparticle recombination noise in a KID.
I show that a KID consisting of a lumped-element resonator can be sensitive enough to be limited by photon noise, which is the fundamental limit for photometry, at a level of optical loading below levels in ground-based CMB experiments.
Finally, I describe an ongoing project to develop multichroic KID pixels that are each sensitive to two linear polarization states in two spectral bands, intended for the next generation of CMB experiments.
I show that a prototype 23-pixel array can detect millimeter-wave light, and present characterization measurements of the detectors.
