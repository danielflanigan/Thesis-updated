\section{The BCS theory and the ground state }
\label{sec:theory.ground_state}

\subsection{The Cooper pair condensate}
\label{sec:theory.ground_state.condensate}

\todo[inline]{Re-read Tinkham and update.}
In the Bardeen-Cooper-Schrieffer (BCS) theory of superconductivity~\autocite{BCS1957PRL, BCS1957PR}, the superconducting ground state can be described in terms of individual-electron (Bloch) states occupied in pairs with opposite momentum and spin, called Cooper pairs~\autocite{Cooper1956PRL}.
These Cooper pairs form due to a phonon-mediated attractive potential $\bcspotential$ between electrons that are within the Debye energy $\phononenergy_\debye$ of the Fermi energy.
The coherence length
$\coherencelength \sim \hbar \velocity_\fermi / \kb \tc$,
where $\velocity_\fermi$ is the Fermi velocity and $\tc$ is the critical temperature for the superconducting phase, corresponds to the minimum size of a Cooper pair as dictated by the uncertainty principle.
For elemental superconductors the coherence length is much greater than the mean spacing between conduction electrons.
While it is more accurate to think of correlations extending over a distance $\coherencelength$, a naive model of the pair condensate is sufficient for the calculations in this thesis.

In a normal metal at temperature $\temperature = 0$, by definition, no states with energy greater than the Fermi energy $\blochenergy_\fermi$ are occupied.
However, in a superconducting metal, even at $\temperature = 0$, some states within an energy range approximately $\kb \tc$ above the Fermi energy remain populated.
The increase in kinetic energy compared to the normal state is outweighed by the decrease in potential energy due to the pairing.

One of the most striking features of the transition to the superconducting state is the Meissner effect, in which, as the temperature is reduced below $\tc$, a screening supercurrent develops to expel any magnetic field from the interior of a bulk superconductor.
This screening is quantified by the penetration depth $\penetrationdepth$, which is the distance from a surface over which the screening supercurrent causes magnetic fields to decay in the bulk.
In the phenomenological London theory, developed long before the BCS theory, the penetration depth at zero temperature is
\begin{equation}
\penetrationdepth_\mathrm{L}
  =
  \left( \frac{\lsvac m_\mathrm{s}}{\impvac n_\mathrm{s} q_\mathrm{s}^2}
  \right)^{1/2},
\end{equation}
where $\lsvac$ is the speed of light in vacuum, $\impvac$ is the impedance of vacuum, and $m_\mathrm{s}$, $n_\mathrm{s}$, and $q_\mathrm{s}$ are respectively the mass, density, and charge of the superconducting carriers.
The aluminum films used for the detectors discussed in this thesis are \SIrange{10}{50}{nm} thick, while the bulk penetration depth for aluminum at low temperature and low frequency is about \SI{50}{nm}~\autocite{Faber1955PRS, Biondi1959bPR}, so the fields completely enter these films. 
The penetration depth is closely related to the reactive part of the superconductor's surface impedance $\impedance_\surface$, discussed in Section~\ref{sec:theory.electrodynamics.surface_impedance} below.

\todo[inline]{Understand time-varying field screening.}
%As in the normal metal, static electric fields are screened over atomic-scale lengths.

In a BCS superconductor below $\tc$, there is a minimum energy $\gap$, called the gap energy, for excitations from the ground state.
These excitations are important for the electrodynamic behavior of a KID, and are discussed in detail later.
In the in the weak-coupling limit of BCS theory, where
$\ssdos \ucvolume \bcspotential \ll 1$,
the critical temperature $\tc$ is proportional to the zero temperature gap energy $\gap_\zerotemp$.
Here, $\ssdos$ is the single spin density of states at the Fermi energy and $\ucvolume$ is the volume of a unit cell.
The relationship is $\gap_\zerotemp = 1.76 \, \kb \tc$, and the numerical factor is accurate to about 20\% in actual elemental superconductors~\autocite{Tinkham2004}.


\subsection{Fiducial parameters}
\label{sec:theory.ground_state.fiducial}

In our current experimental setup we measure the transition temperature either by observing the change in film resistance using a standard four-wire scheme or by observing changes in microwave transmission through the transmission line on a chip containing resonators.
In \SIrange{10}{50}{nm} thick aluminum films we typically measure $\tc$ slightly elevated from the bulk value of \SI{1.2}{K} by \SIrange{0.1}{0.2}{K}, in agreement with other measurements of thin aluminum films~\autocite{Townsend1972PRB}.
Although we are not yet able to directly measure $\gap$, measurements of the gap in thin aluminum films have also shown enhancement above the bulk value~\autocite{Court2008SUST}, and in the absence of a gap measurement we typically assume that the BCS relation remains valid.
Note that some relevant quantities vary exponentially with the ratio of the gap to the temperature, so a small uncertainty in the gap energy may lead to a much larger uncertainty in predictions of such quantities.

As shown in Table~\ref{tab:energies}, the two types of KIDs I discuss here have very different typical resonance frequencies.
In both cases, $\planck \freadout / \gap \ll 1$.
If $\temperature_\bath$ is the bath temperature of the detectors, then for the single-polarization lumped-element KIDs $\planck \freadout_\singlepol / \kb \temperature_\bath \ll 1$, while for the multichroic co-planar waveguide KIDs $\planck \freadout_\multichroic / \kb \temperature_\bath \approx 1$.
This distinction is not practically important for the readout photon occupancy, because the readout power is always sufficiently high to produce an occupancy much greater than one.

\begin{table}[htb]
\centering
\caption
[Fiducial energies, temperatures, and frequencies.]
{
Fiducial energies, temperatures, and frequencies:
$\tc$ is close to the critical temperature we typically measure in aluminum films;
$\temperature_\bath$ is a typical bath temperature; 
$\freadout_\singlepol$ is a typical resonance frequency for the single-polarization lumped-element KIDs used in experiments discussed in Chapters~\ref{chp:loss}~and~\ref{chp:sensitivity};
$\freadout_\multichroic$ is a typical resonance frequency for the multichroic CPW KIDs discussed in Chapter~\ref{chp:multichroic}.
I use these values for numerical estimates, including the slightly elevated gap and critical temperature.
}
\renewcommand{\arraystretch}{1.2}
\begin{tabular}{c S S S S}
\toprule
Parameter & \si{J} & \si{\micro eV} & \si{GHz} & \si{K} \\
\midrule
$\gap_\zerotemp$ & 3.16e-23 & 197.16 & 47.67 & 2.288 \\
$\tc$ & 1.79e-23 & 112.03 & 27.09 & 1.300 \\
$\freadout_\multichroic$ & 1.99e-24 & 12.41 & 3.00 & 0.144 \\
$\temperature_\bath$ & 1.79e-24 & 11.20 & 2.71 & 0.130 \\
$\freadout_\singlepol$ & 6.63e-26 & 0.41 & 0.10 & 0.005 \\
\bottomrule
\end{tabular}
\label{tab:energies}
\end{table}

KIDs have been made from numerous materials, some of which are not well-described by the BCS theory.
However, the KIDs discussed in this work are made either from only aluminum or from both aluminum and niobium, both of which are BCS superconductors.
Throughout this thesis, when making numerical estimates, I use typical material parameters for aluminum and niobium given in Table~\ref{tab:materials} along with the fiducial values given in Table~\ref{tab:energies}, including the slightly elevated values of $\tc$ and $\gap$ that we typically measure.
These should give reasonable descriptions of the detectors we have tested.

\todo[inline]{Check N0 for niobium and understand values given by Kaplan et al.}
\begin{table}[htb]
\centering
\caption[Parameters of superconducting metals used in this thesis.]
{
Parameters of superconducting metals used in this thesis.
See Table~\ref{tab:notation.condensed_matter} for the symbol definitions.
Values are from \textcite{Kaplan1976PRB} except where noted.
}
\renewcommand{\arraystretch}{1.2}
\begin{tabular}{c c c c}
\toprule
Parameter & Unit & Aluminum & Niobium \\
\midrule
$\tc$ (bulk) & \si{K} & 1.19 & 9.2 \\
%$\gap_\zerotemp$ & {} & {} & {} \\ 
$\ssdos$ & \si{eV^{-1}.\micro\meter^{-3}} & \num{1.74e10}~\autocite{Thomas2015SUST} & \num{8.52e10}~\autocite{Jani1988PRB} \\
$\electronphonontime$ & \si{ns} & 438 & 0.149 \\
\bottomrule
\end{tabular}
\label{tab:materials}
\end{table}

\subsection{Radiation detection using superconductors}
\label{sec:theory.ground_state.detection}

For a KID absorbing pair-breaking radiation, the cutoff (lowest detectable) frequency is
\begin{equation}
\foptical_\cutoff 
  =
  2 \gap / \planck
  \approx
  3.5 \, \kb \tc / \planck
  \approx
  \SI{74}{GHz} \left( \tc / \SI{1}{K} \right).
\end{equation}
To minimize the rate of thermal excitations, KIDs must be operated at a low bath temperature.
If $\temperature_\bath$ is the practically achievable bath temperature for a large detector array designed to detect photons with frequency $\foptical$, the superconducting energy gap must satisfy
\begin{equation}
\planck \foptical / 2
  >
  \gap
  \gg
  \kb \temperature_\bath.
\end{equation}
Fortunately, this is currently possible over at least part of the frequency range relevant for CMB observations.
Aluminum can be used to detect pair-breaking photons with frequencies above
$\foptical_\cutoff \approx \SI{100}{GHz}$.
Refrigeration using adiabatic demagnetization or helium dilution allows for cooling of large arrays to temperatures
$\temperature_\bath \approx \SI{0.1}{K} \sim \tc / 10$,
sufficiently low that thermal excitations are negligible.
This allows KIDs to achieve, in principle, the fundamental sensitivity limit set by the statistics of photon arrival.
