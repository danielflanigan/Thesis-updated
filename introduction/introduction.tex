\chapter{Introduction and notation}
\label{chp:introduction}

This thesis deals with the physics and design of sensitive superconducting detectors called kinetic inductance detectors (KIDs).
The detectors discussed here are designed to be used in future experiments to measure the polarization of the cosmic microwave background (CMB).
In Chapter~\ref{chp:cmb}, I give a brief introduction to cosmology, focusing on the the properties of the CMB and on the experiments that measure it.
This chapter is intended to motivate the detector research described in later chapters, and it contains no new results.
In Chapter~\ref{chp:theory}, I introduce kinetic inductance detectors and discuss the basic theory needed to understand their response to light, focusing on the dynamics of the quasiparticle system.
I derive an equation that describes the dynamics of the quasiparticle number, solve it in a simplified form not previously published, and show that it can describe the dynamic response of a detector. 
Chapter~\ref{chp:loss} deals with non-ideal sources of dissipation that can occur in superconducting resonators, degrading their performance as detectors.
I show that magnetic flux vortices in a superconducting thin film can be a significant source of dissipation, and demonstrate some techniques to prevent their formation.
This chapter includes published work (\textcite{Flanigan2016bAPL}) in which we measured the relationship between magnetic field and dissipation due to vortices in a KID.
Chapter~\ref{chp:sensitivity} is concerned with noise sources and KID sensitivity.
Based on the theory presented in Chapter~\ref{chp:theory}, I derive a corrected version of a widely-used equation for the quasiparticle recombination noise in a KID.
I show that a KID consisting of a lumped-element resonator can be sensitive enough to be limited by photon noise, which is the fundamental limit for photometry, at a level of optical loading below levels in ground-based CMB experiments.
This chapter includes published work (\textcite{Flanigan2016aAPL}) in which we measured photon noise using a KID.
Chapter~\ref{chp:multichroic} describes an ongoing project to develop multichroic KID pixels that are each sensitive to two linear polarization states in two spectral bands, intended for the next generation of CMB experiments.
I show that a prototype 23-pixel array can detect millimeter-wave light, and present characterization measurements of the detectors.
This chapter includes material from two papers (\textcite{Johnson2016SPIE, Johnson2017arXiv}) that discuss the results of the project.
In the Appendices, I discuss connections to earlier work, derive some of the equations presented in the main text, and present more information about the hardware used in the experiments.

% Math
\newcommand{\I}{\mathrm{i}}
\newcommand{\E}{\mathrm{e}}
\newcommand{\stepfunction}{\mathrm{H}}

% Subscripts
\newcommand{\quasiparticle}{\mathrm{qp}}
\newcommand{\normal}{\mathrm{n}}
\newcommand{\superconducting}{\mathrm{s}}
\newcommand{\absorbed}{\mathrm{A}}
\newcommand{\incident}{\mathrm{I}}
\newcommand{\source}{\mathrm{S}}
\newcommand{\critical}{\mathrm{c}}
\newcommand{\generation}{\mathrm{G}}
\newcommand{\recombination}{\mathrm{R}}
\newcommand{\thermal}{\mathrm{t}}
\newcommand{\background}{\mathrm{B}}
\newcommand{\resonator}{\mathrm{r}}
\newcommand{\internal}{\mathrm{i}}
\newcommand{\coupling}{\mathrm{c}}
\newcommand{\bath}{\mathrm{b}}
\newcommand{\photon}{\gamma}
\newcommand{\optical}{\mathrm{o}}
\newcommand{\constant}{0}
\newcommand{\tls}{\mathrm{TLS}}
\newcommand{\readout}{\varrho}
\newcommand{\continuouswave}{\mathrm{cw}}
\newcommand{\broadband}{\mathrm{bb}}
\newcommand{\dark}{\mathrm{dark}}
\newcommand{\maximum}{\mathrm{max}}
\newcommand{\knee}{\mathrm{k}}
\newcommand{\cutoff}{\mathrm{c}}
\newcommand{\threshold}{\mathrm{th}}
\newcommand{\ambient}{\mathrm{a}}
\newcommand{\adr}{\mathrm{ADR}}
\newcommand{\magnetarray}{\mathrm{m}}
\newcommand{\fermi}{\mathrm{F}}
\newcommand{\debye}{\mathrm{D}}
\newcommand{\zerotemp}{0}
\newcommand{\surface}{s}
\newcommand{\vacuum}{0}
\newcommand{\multichroic}{\mathrm{mc}}
\newcommand{\singlepol}{\mathrm{1p}}
\newcommand{\dualpol}{\mathrm{2p}}
\newcommand{\kinetic}{\mathrm{k}}
\newcommand{\geometric}{\mathrm{g}}
\newcommand{\vortex}{\mathrm{v}}
\newcommand{\radiation}{\mathrm{rad}}
\newcommand{\substrate}{\mathrm{sub}}
\newcommand{\group}{\mathrm{g}}

% General symbols
\renewcommand{\time}{t}  % L3 Module: l3sys 2015/09/25 v6087 L3 Experimental system/runtime functions
\newcommand{\mass}{m}
\newcommand{\velocity}{v}
\newcommand{\lightspeed}{c}
\newcommand{\soundspeed}{s}
\newcommand{\momentum}{p}
\newcommand{\spectraldensity}{S}
\newcommand{\Rate}{\Gamma}
\newcommand{\rate}{\gamma}
\newcommand{\ssRate}{\overline{\Rate}}
\newcommand{\ssrate}{\overline{\rate}}
\newcommand{\eventrate}{\kappa}
\newcommand{\energy}{E}
\newcommand{\power}{P}
\newcommand{\temperature}{T}
\newcommand{\wvec}{k}
\newcommand{\vwvec}{\vb{\wvec}}
\newcommand{\foptical}{\nu}
\newcommand{\freadout}{f}
\newcommand{\faudio}{\varphi}
\newcommand{\efficiency}{\eta}
\newcommand{\volume}{\mathcal{V}}
\newcommand{\nep}{\mathrm{NEP}}
\newcommand{\spectralindex}{\alpha}
\newcommand{\opticalbandwidth}{B}
\newcommand{\impedance}{Z}
\newcommand{\resistance}{R}
\newcommand{\reactance}{X}
\newcommand{\inductance}{L}
\newcommand{\capacitance}{C}
\newcommand{\efield}{E}
\newcommand{\bfield}{B}
\newcommand{\dfield}{D}
\newcommand{\hfield}{H}
\newcommand{\vefield}{\vb{E}}
\newcommand{\vbfield}{\vb{B}}
\newcommand{\vdfield}{\vb{D}}
\newcommand{\vhfield}{\vb{H}}
\newcommand{\voltage}{V}
\newcommand{\currentdensity}{J}
\newcommand{\current}{I}
\newcommand{\quantum}{k}
\newcommand{\resistivity}{\rho}
\newcommand{\conductivity}{\sigma}
\newcommand{\diffusion}{D}

% Physical constants
\newcommand{\lsvac}{\lightspeed_\vacuum}
\newcommand{\impvac}{\impedance_\vacuum}
\newcommand{\planck}{h}
\newcommand{\kb}{k_\mathrm{B}}
\newcommand{\unitcharge}{e}
\newcommand{\fluxquantum}{\Phi_0}

\begin{table}[tb]
\centering
\caption
{Physical constants.}
\renewcommand{\arraystretch}{1.2}
\begin{tabular}{c l}
\toprule
Symbol & Meaning \\
\midrule
$\lsvac$ & The speed of light in vacuum \\ 
$\impvac$ & The impedance of vacuum \\
$\planck$ & The Planck constant \\
$\hbar$ & The reduced Planck constant, $\planck / 2 \pi$ \\
$\kb$ & The Boltzmann constant \\
$\unitcharge$ & The elementary charge (positive) \\
$\fluxquantum$ & The superconducting flux quantum \\
\bottomrule
\end{tabular}
\label{tab:notation.physical_constants}
\end{table}

% Cosmology and astrophysics
\newcommand{\scalefactor}{a}
\newcommand{\redshift}{z}
\newcommand{\wavelength}{\lambda}

% Miscellaneous symbols
\newcommand{\photonoccupancy}{n}
\newcommand{\detectiontime}{\tau}
\newcommand{\tracewidth}{w}
\newcommand{\tracelength}{\ell}
\newcommand{\thickness}{d}
\newcommand{\distance}{d}
\newcommand{\vortexnumber}{N}
\newcommand{\forwardscattering}{S_{21}}
\newcommand{\amplifierwhite}{A}
\newcommand{\detectorwhite}{W}
\newcommand{\detectorred}{R}
\newcommand{\responseqpoccupancy}{K}
\newcommand{\normresponse}{\Upsilon}
\newcommand{\ssnormresponse}{\overline{\normresponse}}
\newcommand{\jonasbeta}{\beta}

\begin{table}[tb]
\centering
\caption
{General symbols.}
\renewcommand{\arraystretch}{1.2}
\begin{tabular}{c l}
\toprule
Symbol & Meaning \\
\midrule
$\Rate$ & A macroscopic (extensive) rate of some process in a given volume \\
$\rate$ & A microscopic (intensive) rate per unit volume \\
$\foptical$ & An ``optical'' frequency, used for millimeter-wave light around \SI{100}{GHz} \\
$\freadout$ & A microwave frequency, used for readout tones around \SI{1}{GHz} \\
$\faudio$ & An ``audio'' frequency, used for detector time-ordered data around \SI{1}{kHz} \\
\bottomrule
\end{tabular}
\label{tab:notation.general}
\end{table}

% Condensed matter: solids, phonons, and superconductivity
\newcommand{\gap}{\Delta}
\newcommand{\tc}{\temperature_\critical}
\newcommand{\coherencelength}{\xi_0}
\newcommand{\penetrationdepth}{\lambda}
\newcommand{\qpnumber}{N_\quasiparticle}
\newcommand{\ssqpnumber}{\overline{N}_\quasiparticle}
\newcommand{\qpdensity}{n_\quasiparticle}
\newcommand{\ssqpdensity}{\overline{n}_\quasiparticle}
\newcommand{\phononenergy}{\Omega}
\newcommand{\blochenergy}{\varepsilon}
\newcommand{\blochenergyf}{\xi}
\newcommand{\qprdos}{\rho}
\newcommand{\ssdos}{N_0}
\newcommand{\supssdos}{N_\mathrm{s}}
\newcommand{\bcspotential}{V_\mathrm{BCS}}
\newcommand{\qprecombination}{R}
\newcommand{\qprecombinationeff}{\mathcal{R}}
\newcommand{\qpsingledecay}{\mathcal{S}}
\newcommand{\qprecombinationtime}{\tau_\mathrm{R}}
\newcommand{\qprelaxationtime}{\tau_\quasiparticle}
\newcommand{\qpphononscatteringtime}{\tau_\mathrm{s}}
\newcommand{\electronphonontime}{\tau_0}
\newcommand{\phononpairbreakingtime}{\tau_\mathrm{br}}
\newcommand{\phononescapetime}{\tau_\mathrm{es}}
\newcommand{\phonontrapping}{F}
\newcommand{\meanfreepath}{\ell}
\newcommand{\qpoccupancy}{\mathcal{F}}
\newcommand{\ssqpoccupancy}{\overline{\qpoccupancy}}
\newcommand{\reconductivity}{\conductivity_1}
\newcommand{\imconductivity}{\conductivity_2}
\newcommand{\ssreconductivity}{\overline{\conductivity}_1}
\newcommand{\ssimconductivity}{\overline{\conductivity}_2}
\newcommand{\normalconductivity}{\conductivity_\normal}
\newcommand{\dynes}{\Gamma}
\newcommand{\mitrovic}{\Delta_2}
\newcommand{\ucvolume}{\volume_\mathrm{uc}}
\newcommand{\qpdiffusion}{\diffusion_\quasiparticle}
\newcommand{\normaldiffusion}{\diffusion_\normal}

\begin{table}[tb]
\centering
\caption
{Symbols related to condensed matter: solids, superconductivity, and phonons.}
\renewcommand{\arraystretch}{1.2}
\begin{tabular}{c l}
\toprule
Symbol & Meaning \\
\midrule
$\gap \; (\gap_\zerotemp)$ & The superconductor gap energy (at zero temperature) \\
$\tc$ & The critical temperature of a superconductor \\
$\coherencelength$ & The superconducting coherence length \\
$\penetrationdepth$ & The superconducting penetration depth \\
$\qpnumber$ & The number of quasiparticles in a given region  \\
$\qpdensity$ & The number of quasiparticles per unit volume \\
$\phononenergy$ & The energy of a phonon \\
$\blochenergy$ & The energy of a Bloch state \\
$\blochenergy_\fermi$ & The Fermi energy \\
$\blochenergyf$ & The energy of a Bloch state, measured from the Fermi energy \\
$\velocity_\fermi$ & The Fermi velocity \\
$\qprdos$ & The reduced quasiparticle density of states \\
$\ssdos$ & The single-spin density of electron states at the Fermi energy \\
$\supssdos$ & The single-spin density of quasiparticle states \\
$\bcspotential$ & The BCS potential energy \\
$\qprecombination$ & The intrinsic quasiparticle recombination constant \\
$\qprecombinationeff$ & The effective quasiparticle recombination constant, including phonon trapping \\
$\qpsingledecay$ & The single-quasiparticle decay constant \\
$\qprecombinationtime$ & The average recombination lifetime of a single quasiparticle \\
$\qprelaxationtime$ & The relaxation time of a small perturbation to the quasiparticle density \\
$\electronphonontime$ & The characteristic electron-phonon interaction time \\
$\qpphononscatteringtime$ & The quasiparticle-phonon scattering time \\
$\phononpairbreakingtime$ & The phonon pair-breaking time \\
$\phononescapetime$ & The phonon escape time from a film \\
$\phonontrapping$ & The phonon trapping factor \\
$\meanfreepath$ & The electron mean-free path \\
$\qpoccupancy$ & The quasiparticle occupancy (``distribution function'') \\
$\normalconductivity$ & The conductivity in the normal state just above $\tc$ \\
$\reconductivity$ & The real part of the complex conductivity \\
$\imconductivity$ & The imaginary part of the complex conductivity \\
$\dynes$ & The imaginary part of the quasiparticle energy \\
$\mitrovic$ & The imaginary part of the gap energy \\
$\ucvolume$ & The volume of a unit cell in a crystal \\
\bottomrule
\end{tabular}
\label{tab:notation.condensed_matter}
\end{table}

% Resonator symbols
\newcommand{\shift}{s}
\newcommand{\ssshift}{\overline{\shift}}
\newcommand{\detuning}{x}
\newcommand{\ssdetuning}{\overline{\detuning}}
\newcommand{\loss}{\Lambda}
\newcommand{\ssloss}{\overline{\loss}}
\newcommand{\qf}{Q}
\newcommand{\asymmetry}{A}
\newcommand{\surfimpexp}{\zeta}
\newcommand{\kifraction}{\alpha}
\newcommand{\qpfraction}{\chi_\quasiparticle}
\newcommand{\restransfer}{\xi_\resonator}
\newcommand{\pbefficiency}{\efficiency_\mathrm{pb}}
\newcommand{\qpperphoton}{q}
\newcommand{\adiabatici}{\Sigma_{\loss_\internal}}
\newcommand{\adiabaticx}{\Sigma_\detuning}

\begin{table}[tb]
\centering
\caption
{Symbols related to resonators and kinetic inductance detectors.}
\renewcommand{\arraystretch}{1.2}
\begin{tabular}{c l}
\toprule
Symbol & Meaning \\
\midrule
$\freadout_\resonator$ & The resonance frequency \\
$\freadout_\readout$ & The readout tone frequency \\
$\shift$ & The fractional resonance frequency shift from the fiducial, or zero temperature, case \\
$\detuning$ & The fractional detuning of the resonance frequency from the readout frequency \\
$\qf$ & A resonator quality factor: $\qf_\alpha \equiv \loss_\alpha^{-1}$ for all subscripts $\alpha$ \\
$\loss$ & A resonator inverse quality factor, or loss: $\loss_\alpha \equiv \qf_\alpha^{-1}$ for all subscripts $\alpha$ \\
$\asymmetry$ & A parameter that quantifies the asymmetry of a resonance \\
$\surfimpexp$ & The exponent in the dependence of the surface impedance on film thickness \\
$\kifraction$ & The effective kinetic inductance fraction \\
$\qpfraction$ & The ratio of the quasiparticle loss to the total loss \\
$\restransfer$ & The frequency-dependent resonator transfer function \\
$\pbefficiency$ & The photon pair-breaking efficiency \\
$\qpperphoton$ & The average number of quasiparticles excited per absorbed photon \\
\bottomrule
\end{tabular}
\label{tab:notation.resonator}
\end{table}

Tables~\ref{tab:notation.physical_constants},~\ref{tab:notation.general},~\ref{tab:notation.condensed_matter},~and~\ref{tab:notation.resonator} present the important symbols.
I have attempted to define all symbols where they are first used in the text.
In many places I use an over-bar to denote a steady-state quantity that does not vary in time, and a $\delta$ prefix to denote the time-dependent difference from the steady-state value.
For example, 
$\delta\qpnumber(\time) = \qpnumber(\time) - \ssqpnumber$
denotes a time-dependent deviation from the steady-state number of quasiparticle excitations in a superconductor.
Except where noted, I use SI units.
